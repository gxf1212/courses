\chapter*{编写说明}
\markboth{Introduction}{编写说明}

随着科技的进步,各种仪器逐渐成为科学研究中必不可少的工具。为了探寻微观世界的奥秘,精确探测物质结构、含量、分布,获取有关研究体系的信息,研究人员都必须借助仪器这只“眼睛”。正是因为有了更精良的仪器,很多领域的科学问题才得以逐渐被科学家解决。可以说,没有仪器,就没有如今的科学成果。

《现代仪器分析》是化生试验班的一门专业核心课程,占2学分,讲授共32学时,由生命学院开设。课程涉及的主要学科是\textbf{分析化学},旨在帮助同学们掌握化学和生物研究中常用也是基础的仪器的原理、结构、适用范围等,为将来进入实验室进行实际操作打下理论基础。
这些分析手段的主要目的有二:检测(定性、定量)和分离,几乎每一章节在两方面都有所涉及。不过,各章节之间并没有紧密的联系,几乎是可以独立进行学习的。这些检测手段相互补充,共同构成了这门课的知识体系。

本课程的期末考试占总成绩的$70\%$,所以期末综合复习显得比较重要。为了明确课程要求同学掌握的内容,授课教师提供了一份提纲,为同学指明了需要复习的重点内容。为了方便同学们复习,化生81班班委组织了一些成绩优秀的同学梳理了提纲涉及的具体知识点,并用\LaTeX 整理成书。课程内容主要分为八章,分别列于下方框中。
\begin{tcolorbox}[title={\bfseries 编写组成员}]
	\ding{228} 张天翊:第一章\hspace{1em}\textbf{色谱}\\
	\ding{228} 王炜喆:第二章\hspace{1em}\textbf{质谱}\\
	\ding{228} 何琦璟;第三章\hspace{1em}\textbf{紫外-可见吸收光谱}\\
	\ding{228} 张宇博:第四章\hspace{1em}\textbf{分子发光分析}\\
	\ding{228} 郭骐瑞:第五章\hspace{1em}\textbf{红外光谱和拉曼光谱}\\
	\ding{228} 程肖然:第六章\hspace{1em}\textbf{核磁共振波谱}\\
	\ding{228} 高旭帆:第七章\hspace{1em}\textbf{电化学分析}\\
	\ding{228} 刘祎宁:第八章\hspace{1em}\textbf{原子光谱}\\
	\ding{228} 高旭帆、郭骐瑞:\textbf{排版、设计、整理}
\end{tcolorbox}
在此要对组织、编写和排版人员表示衷心的感谢!

在这份提纲中,有大量知识点需要记忆,希望同学们根据提纲有侧重点地复习。同时,这是化生试验班的第一份集体复习资料,各位编写者、排版者的风格可能有一定差异,内容、细节也存在可能的疏漏,必要时读者可以查阅课本、课件等其他资料。欢迎大家对这份提纲提出建议,也更希望大家多多支持我们的作品!

最后,我们希望这份提纲能帮助同学们取得一个理想的成绩!

\vskip 1.5cm

\begin{flushright}
	化生81 期末资料编写组\\
	2020\ 年\ 5\ 月\ 17\ 日
\end{flushright}