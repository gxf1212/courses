\documentclass[cn,11pt,chinese]{elegantbook}

\title{现代仪器分析}
\subtitle{翻译项目介绍}
%\bioinfo{整理}{化生81}
\date{二〇二〇年四月}



\extrainfo{}

\logo{xjtu.png}
\cover{plastic.png}

% 本文档命令
\usepackage{array}
\newcommand{\ccr}[1]{\makecell{{\color{#1}\rule{1cm}{1cm}}}}

\begin{document}

\maketitle
\frontmatter


\markboth{Introduction}{前言}
\vskip 1.5cm

\tableofcontents

\mainmatter
\chapter{排版格式介绍}
\begin{introduction}
    \item 概念及定义
    \item 公式
    \item 笔记
    \item 例子
\end{introduction}

\subsection{概念及定义}
排版过程中的定义我们使用如下格式
\begin{definition*}{概念及定义}{defin}
    
\end{definition*}

\subsection{公式}
一些比较重要的公式我们使用如下格式
\begin{theorem*}{公式/定理}{theo}
    
\end{theorem*}

\subsection{笔记}
\note 需要补充、解释的我们加在这里

\subsection{例子}
\begin{example}
    举例我们放在这里
\end{example}
\end{document}