\newpage
\hypersetup{pageanchor=true}
\tcbset{colframe = structurecolor, colback = white}

\begin{tcolorbox}[title={\bfseries 作品信息}]
	\makeatletter
	\ding{228} \textbf{标题:}\@title{} - \textit{Review Outline on Modern Instrumental Analysis}\\
	\ding{228} \textbf{作者:}\@author\\
	\ding{228} \textbf{出品时间:}\@date\\
	\ding{228} \textbf{总页数:}\pageref{LastPage}
	\makeatother
\end{tcolorbox}

\begin{tcolorbox}[title={\bfseries 关于本提纲用途的说明}]
\ding{228} 主要用于化生试验班的备考,并且会在西安交通大学范围内共享;\\
\ding{228} \texttt{这份提纲旨在助力备考,内容比较精简,供理解的内容较浅,习题也不充足,所以对于希望补充专业知识为将来所用的同学可能是不够的,只能当做入门的读物。如果希望深入学习,可以参考最新的教材或网络课程资源;}\\
\ding{228} 如果将来有关专业的同学需要参考这份提纲,可能需要考虑到课程大纲的调整。
\ding{228} 如希望补充本提纲的内容,或认为笔记有错漏或待完善之处,欢迎联系资料负责人。
\begin{itemize}
	\item 资料负责人:化生81\hspace{1em}高旭帆
	\item \faEnvelopeOpen ~~ 邮箱:\texttt{gxf1212@stu.xjtu.edu.cn}
\end{itemize}
\end{tcolorbox}

\begin{tcolorbox}[title={\bfseries 主要参考资料}]
\ding{228} 刘密新等. 仪器分析(第二版)[M]. 北京:清华大学出版社, 2002\\
\ding{228} 三位老师的课件。在此要向三位授课教师:魏晶老师、武亚艳老师、赵永席老师表示衷心的感谢!\\
\ding{228} 一些网络资源
\end{tcolorbox}


\begin{tcolorbox}[title={\bfseries 许可证说明}]
\centerline{\tcbox{\fangsong 知识共享 (Creative Commons) BY-NC-ND 4.0 协议}}
本作品采用 \href{https://creativecommons.org/licenses/by-nc-nd/4.0/}{\textbf{CC协议}}
进行许可。使用者可以在给出作者署名及资料来源的前提下对本作品进行转载,
但不得对本作品进行修改,亦不得基于本作品进行二次创作,
不得将本作品运用于商业用途。
\end{tcolorbox}

\begin{tcolorbox}
	本作品已发布于GitHub之上,发布地址为:\\
	\centerline{\url{https://qyxf.site/bookhub}}
	本作品的版本号为\textsf{v1.0}。
	
	本提纲使用了Elegant\LaTeX 模板进行排版,并进行了微小的调整,在此也要感谢模板的制作者。
\end{tcolorbox}



%\newpage
